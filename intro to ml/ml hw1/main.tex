 \documentclass[12pt]{article}
\usepackage[utf8]{inputenc}
\usepackage[top=1in, bottom=0.75in, left=0.75in, right=0.75in, headheight=15pt]{geometry}
\usepackage{amsmath, amssymb, amsthm, graphicx, hyperref, enumerate, multirow,  multicol, tikz, centernot, cancel, forest, lipsum, mathtools, bm, esvect, fancyhdr, esdiff, float, parskip, comment}

\DeclareMathSymbol{*}{\mathbin}{symbols}{"01} % change * to /cdot inside math
% \begingroup % let only this align, etc. break across pages
% \allowdisplaybreaks
% \begin{align}
%     ....
% \end{align}
% \endgroup

% \texorpdfstring{$k$}{k} math inside (sub)section label

\pagestyle{fancy}
\fancyhead[L]{Liheng Cao}
% \fancyhead[C]{center}
\fancyhead[R]{lc4241}


\title{Title}
\author{Liheng Cao}
% \date{Date}

\begin{document}
\maketitle

\section{}
\subsection*{(a)}
This is regression because salary is a continuous variable. $N = 500, d = 3$

\subsection*{(b)}
The is classification because the 2 possible outcomes are ``success" or ``failure". $N = 20, d = 14$
\newpage

\section{}
\subsection*{(a)}
\begin{enumerate}
	\item Given previous stock history, we could classify whether to buy, sell, or hold (not do anything) the stock. The features would be previous stock prices, and the target would be buy, sell, or hold.
	
	\item 
\end{enumerate}
\subsection*{(b)}
\begin{enumerate}
	\item 
\end{enumerate}
\newpage

\section{}
\subsection*{(a)}




\end{document}


