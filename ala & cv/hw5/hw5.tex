\documentclass[12pt]{article}
\usepackage[utf8]{inputenc}
\usepackage[top=0.75in, bottom=0.75in, left=0.75in, right=0.75in, headheight=15pt]{geometry}
\usepackage{amsmath, amssymb, amsthm, graphicx, hyperref, enumerate, multirow,  multicol, tikz, centernot, cancel, forest, lipsum, mathtools, bm, esvect, fancyhdr, esdiff, float, parskip, comment}

\DeclareMathSymbol{*}{\mathbin}{symbols}{"01} % change * to /cdot inside math
% \begingroup % let only this align, etc. break across pages
% \allowdisplaybreaks
% \begin{align}
%     ....
% \end{align}
% \endgroup

% \texorpdfstring{$k$}{k} math inside (sub)section label

\pagestyle{fancy}
\fancyhead[L]{Liheng Cao}
% \fancyhead[C]{center}
\fancyhead[R]{}

\title{Homework 5}
\author{Liheng Cao}
% \date{Date}

\begin{document}
\maketitle

\section{}
\begin{enumerate}[(a)]
	\item Rewrite as \[ \dfrac{z^2+2}{z-1} \] The singularity is at $ z=1 $, the order is 1, and the residue is $ 1^2 + 2 = 3 $.
	
	\item Rewrite as \[ \dfrac{z^3/2^3}{(z+1/2)^3} \] The singularity is at $ z=-1/2 $, the order is 3, and the residue is $ \diff*[2]{\left(z^3/8\right)}{z}{-1/2} * 1/2! = 6*-1/2 / 8 * 1/2 = -3/16$
	
	\item Rewrite as \[ \dfrac{\exp{(z)}/(z+i\pi)}{z-i\pi}, \dfrac{\exp{(z)}/(z-i\pi)}{z+i\pi} \] The poles are at $ z = \pm i\pi $, the orders are both 1, and the residues are $ \dfrac{\exp{(i\pi)}}{2i\pi}, \dfrac{\exp{(-i\pi)}}{-2i\pi} = \dfrac{1}{2i\pi}, \dfrac{1}{-2i\pi} = \dfrac{-i}{2\pi}, \dfrac{i}{2\pi} = \mp \dfrac{i}{2\pi}$
	
	\item Rewrite as \[ \dfrac{f(z)}{g(z)} = \dfrac{1}{\sin{(z)}} \longrightarrow \dfrac{f(z)}{g'(z)} = \dfrac{1}{\cos{(z)}} \] The singularities are located at $ \pi n, n \in \mathbb{Z} $, the order is 1, and the residue is $ (-1)^n $
\end{enumerate}
\newpage

\section{}
\subsection{f(z)}
\begin{enumerate}[(a)]
	\item 
	\begin{align*}
		z^3 \sin{(1/z)} &= z^3 \left(\dfrac{1}{z} - \dfrac{1}{z^3 * 3!} + \dfrac{1}{z^5  * 5!} + \cdots + (-1)^n \dfrac{1}{z^{2n+1} * (2n+1)!}\right)\\
		&= z^2 - \dfrac{1}{3!} + \dfrac{1}{z^2} + \cdots + (-1)^n\dfrac{1}{z^{2n-2} * (2n+1)!}
	\end{align*}
	The principal part consists of the terms with negative powers of $ z $. In this case, it is \[ \dfrac{1}{z^2} + \cdots + (-1)^n\dfrac{1}{z^{2n-2} * (2n+1)!} \]
	
	\item The residual is 0 because the coefficient of the $ z^{-1} $ term is 0. 
	
	\item Since there are an infinite amount of negatives power terms, this is an essential singularity.
\end{enumerate}

\subsection{g(z)}
\begin{enumerate}[(a)]
	\item 
	\begin{align*}
		\dfrac{1}{z^4}e^z &= \dfrac{1}{z^4} \left(1 + z + \dfrac{z^2}{2!} + \dfrac{z^3}{3!} + \dfrac{z^4}{4!} + \cdots + \dfrac{z^n}{n!}\right)\\
		&= \dfrac{1}{z^4} + \dfrac{1}{z^3} + \dfrac{1}{z^2 *  2!} + \dfrac{1}{z  * 3!} + \dfrac{1}{4!} +\cdots+ \dfrac{z^{z-4}}{n!}
	\end{align*}
	The principal part is \[ \dfrac{1}{z^4} + \dfrac{1}{z^3} + \dfrac{1}{z^2 *  2!} + \dfrac{1}{z  * 3!} \]
	
	\item The residue is $ 3! $.
	
	\item This is a pole of order 4 because there are 4 terms in the principal part (4 terms with negative powers).
\end{enumerate}
\newpage

\section{}
\begin{enumerate}[(a)]
	\item $ R = |a| $ because the point $ a $ is $ |a| $ away from $ z=0 $. We use the Taylor expansion theorem.
	
	\item \[ a_n = \diff[n]{1/z}{z} / n! \]
	\begin{align*}
		\dfrac{1}{z} &= a_0 (z-a)^0 / 0! + a_1 (z-a)^1 / 1! + a_2 (z-a)^2 / 2! + \cdots + a_n (z-a)^n / n! \\
		&= 1/a - \dfrac{1}{a^2} (z-a) + \dfrac{1}{a^3} (z-a)^2 + \cdots + a_n (z-a)^n / n!
	\end{align*}

	\item 
	\begin{align*}
		\dfrac{1}{z} &= \dfrac{1}{a} * \dfrac{1}{1- (-(z-a)/a)}\\
		&= \dfrac{1}{a} * \left((-(z-a)/a)^0 + (-(z-a)/a)^1 + (-(z-a)/a)^2 + \cdots + (-(z-a)/a)^n\right)\\
		&= \dfrac{1}{a} - \dfrac{z-a}{a^2} + \dfrac{(z-a)^2}{a^3} +\cdots+  \dfrac{(z-a)^n}{a^{n+1}}
	\end{align*}
\end{enumerate}
\newpage

\section{}
Let's find the distances of the singularities to each point first. The distance from $ 1+i $ to $ +i $ is 1. The distance from $ 1+i $ to $ -i $ is $ \sqrt{5} \approx 2.24 $. The distance from $ 1+i $ to $ \sqrt{3}i $ is $ ~1.24 $. The distance from $ 1+i $ to $ -\sqrt{3}i $ is $ ~2.91 $. 
\begin{enumerate}[(a)]
	\item The integral evaluates to 0 because it has no singularities inside its bounds. (Cauchy-Goursat Theorem)
	
	\item The singularities are at $ z = i, \sqrt{3}i $. 
	\[ \operatorname{Res}\limits_{z=i} \dfrac{z^3/ \left[(z+i)(z^2 + 3)\right]}{z-i} = i^3 / \left(2i * (-1 + 3)\right) = 1 / 4\]
	\[ \operatorname{Res}\limits_{z=\sqrt{3}i} \dfrac{z^3/ \left[(z^2+1)(z + \sqrt{3}i)\right]}{z-\sqrt{3}i} = 3/4 \]
	
	The integral is equal to $ 2\pi i (3/4 + 1/4) = 2\pi i$
	
	\item The singularities are those of the previous integral, plus the ones located at $ z = -i, -\sqrt{3i} $
	\[ \operatorname{Res}\limits_{z=-i} \dfrac{z^3/ \left[(z-i)(z^2 + 3)\right]}{z+i} = -1/4 \]
	\[ \operatorname{Res}\limits_{z=-\sqrt{3}i} \dfrac{z^3/ \left[(z^2+1)(z - \sqrt{3}i)\right]}{z+\sqrt{3}i} = 3/4\]
	
	The integral is equal to the previous integral plus $ 2\pi i (3/4-1/4) = \pi i $. The result is $ 3\pi i $
	
	\item Since all the singularities have been accounted for, the result is the same as (b): $ 3\pi i $
	
	\item Same situation as (c): $ 3\pi i $	
\end{enumerate}
\newpage

\section{}
These functions are even, so \[ \int_{0}^{\infty} \cdots = \dfrac{1}{2}\int_{-\infty}^{\infty} \cdots  \]

\section{}
There is only 1 singularity with $ Im(z) > 0 $ at $ z = i $ 
\[\operatorname{Res}\limits_{z=i} \dfrac{1/(z+i)}{z-i} = 1/2i = -i/2\]
The integral would be equal to $ 1/2 * 2\pi i * -i/2 = \pi/2 $

\section{}
There is only 1 singularity with $ Im(z) > 0 $ at $ z = i $ 
\[\operatorname{Res}\limits_{z=i} \dfrac{1/(z+i)^2}{(z-i)^2} = \diff{(1/(z+i)^2)}{z}(i) = -i/4 \]
The integral is equal to $ 1/2 * 2\pi i * -i/4 = \pi/4 $

\section{}
The eligible singularities are at $ z = \sqrt{i}, z = -\sqrt{-i} = -\dfrac{1}{\sqrt{2}} + \dfrac{1}{\sqrt{2}i}$.
\[\operatorname{Res}\limits_{z=\sqrt{i}} \dfrac{1/\left((z+\sqrt{i})(z^2 + i)\right)}{z-\sqrt{i}} = \dfrac{-\sqrt{2}}{8} - \dfrac{\sqrt{2}}{8}i \]
\[\operatorname{Res}\limits_{z=-\sqrt{-i}} \dfrac{1/\left((x-\sqrt{-i})(z^2 - i)\right)}{z+\sqrt{-i}} = \dfrac{\sqrt{2}}{8} - \dfrac{\sqrt{2}}{8}i \]
The integral is equal to $ 1/2 * 2\pi i * (\cancel{\dfrac{-\sqrt{2}}{8}} - \dfrac{\sqrt{2}}{8}i + \cancel{\dfrac{\sqrt{2}}{8}} - \dfrac{\sqrt{2}}{8}i) = \pi \sqrt{2}/4 = \dfrac{\pi}{2\sqrt{2}}$

\section{}
The eligible singularities are at $ z = i, 2i $
\[\operatorname{Res}\limits_{z=i}  \dfrac{z^2/\left((z+i)(z^2 + 4)\right)}{z-i} = i/6 \]
\[\operatorname{Res}\limits_{z=2i}  \dfrac{z^2/\left((z^2+1)(z + 2i)\right)}{z-2i} = -i/3 \]
The integral is $ 1/2 * 2\pi i * (i/6 - i/3) = \pi i *-i/6 = \pi/6 $

\section{}
The eligible singularities are at $ z=2i, 3i $
\[\operatorname{Res}\limits_{z=2i}  \dfrac{z^2/\left((z+2i)^2 (z^2 + 9)\right)}{(z-2i)^2} = -13i/200 \]
\[\operatorname{Res}\limits_{z=3i}  \dfrac{z^2/\left((z^2 + 4)^2(z+3i)\right)}{z-3i} = 3i/50 \]
The integral is equal to $ 1/2 * 2\pi i *(-13i/200 + 3i/50) = \pi i * -i/200 = \pi/200$
\newpage

\section{}
\begin{enumerate}[(a)]
	\item 
		\begin{align*}
			\cdots &= \oint \dfrac{1}{5+4\left(\dfrac{z+z^{-1}}{2}\right)} \dfrac{dz}{iz}\\
			&= \oint \dfrac{1}{5+2z+2z^{-1}} \dfrac{dz}{iz}\\
			&= \oint \dfrac{-i}{2z^2+5z+2} dz\\
		\end{align*}
		Find the singularities:
		\begin{align*}
			2z^2 +5z + 2 = 0\\
			\dfrac{-5 \pm \sqrt{25-4*2*4}}{4}\\
			\dfrac{-5 \pm 3}{4}\\
			-2, -1/2\\
			=(2z+1)(z+2)
		\end{align*}
		Find the residuals:
		\begin{align*}
			\operatorname{Res}_{z=-2} \dfrac{-i/(2z+1)}{z+2} = i/3
			\operatorname{Res}_{z=-1/2} \dfrac{-i/(z+2)}{2z+1} = -2i/3
		\end{align*}
		The integral is equal to $ 2\pi i * (i/3 + -2i/3) = 2\pi i * -i/3 = 2\pi/3 $

	\item 
		\begin{align*}
			\cdots &= \oint \dfrac{1}{1+\left(\dfrac{z-z^{-1}}{2i}\right)^2} \dfrac{dz}{iz}\\
			&= \oint \dfrac{1}{1+\dfrac{z^2 - 2 + z^{-2}}{-4}} \dfrac{dz}{iz}\\
			&= \oint \dfrac{-4/i}{-4+z^2 - 2 + z^{-2}} dz/z\\
			&= \oint \dfrac{4i}{z^3 -6z +z^{-1}}
		\end{align*}
		Find the singularities:
		\begin{align*}
			z^3 -6z +z^{-1} &= 0\\
			z^{4} - 6z^{2} + 1 &= 0\\
			a^2 - 6a + 1 &= 0 \tag{$ a = z^2 $}\\
			\dfrac{6\pm\sqrt{36-4*1*1}}{2}\\
			\dfrac{6\pm\sqrt{32}}{2}\\
			\dfrac{6 \pm 4 \sqrt{2}}{2}\\
			a &= 3\pm 2\sqrt{2}\\
			z &= \pm (\sqrt{2} \pm 1)
		\end{align*}
		Only $ \sqrt{2}-1, -\sqrt{2}+1 $ are inside out unit circle. Calculate residues:
		\begin{align*}
			\operatorname{Res}_{z=\sqrt{2} - 1} \dfrac{-4/i}{z^3-6z+z^{-1}} = \dfrac{-4/i}{3z^2-6-z^{-2}}\Bigr|_{z=\sqrt{2} + 1} = -\dfrac{i}{2\sqrt{2}}
			\operatorname{Res}_{z=-\sqrt{2} + 1} \dfrac{-4/i}{z^3-6z+z^{-1}} = \dfrac{-4/i}{3z^2-6-z^{-2}}\Bigr|_{z=-\sqrt{2} + 1} = -\dfrac{i}{2\sqrt{2}}
		\end{align*}
		The integral is equal to $ 2\pi i * (-\dfrac{i}{2\sqrt{2}} -\dfrac{i}{2\sqrt{2}}) = 2\pi i *(-\dfrac{i}{\sqrt{2}}) = 2\pi /\sqrt{2} = \sqrt{2}\pi$
\end{enumerate}

\end{document}


