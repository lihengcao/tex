\documentclass[12pt]{article}
\usepackage[utf8]{inputenc}
\usepackage[top=1in, bottom=0.75in, left=0.75in, right=0.75in, headheight=15pt]{geometry}
\usepackage{amsmath, amssymb, amsthm, graphicx, hyperref, enumerate, multirow,  multicol, tikz, centernot, cancel, forest, lipsum, mathtools, bm, esvect, fancyhdr, esdiff, float, parskip, comment}

\DeclareMathSymbol{*}{\mathbin}{symbols}{"01} % change * to /cdot inside math
% \begingroup % let only this align, etc. break across pages
% \allowdisplaybreaks
% \begin{align}
%     ....
% \end{align}
% \endgroup

% \texorpdfstring{$k$}{k} math inside (sub)section label


\title{hw2}
\author{Liheng Cao}
% \date{Date}
\pagestyle{fancy}
\fancyhead[L]{Liheng Cao}
% \fancyhead[C]{center}
%\fancyhead[R]{lc4241}
\fancyhead[R]{MA 3113}


\begin{document}
\maketitle

\section{}\hrule
\subsection{a)}
\begin{align*}
	e^z &= e^{x+iy}\\
	&= e^x e^{iy}\\
	&= e^x \left( \cos{(y)} + i\sin{(y)}\right) \\
	&= e^x \cos{(y)} + i e^x \sin{(y)}\\
\end{align*}

\boxed{X = e^x \cos{(y)}, Y = e^x \sin{(y)}}

\subsection{b)}

Let $z = x+iy, w = a+ib$. Addition and multiplication are commutative, so we can rearrange to get the real together, and the imaginaries together.
\begin{align*}
	e^z * e^w &= e^{a+ib} * e ^{x+iy}\\
	&= e^a * e^ib * e^x * e^iy\\
	&= e^{ax} * e^{i(b+y)}\\
	&= e^{a + ib + x + iy}\\
	&= e^{z+w}\\
	&\qed
\end{align*}

\subsection{c)}
Show $e^{2\pi i n} = +1$
\begin{align*}
	e^{2\pi i n} &= \left( e^{2\pi i}\right)^n\\
	&= \left( \cos{2\pi} + i\sin{2\pi}\right)^n\\
	&= \left( 1  i * 0\right)^n\\
	&= 1^n\\
	&= +1\\
	&\qed
\end{align*}

Show $e^z$ is periodic with period $2\pi i$: $e^{z+2\pi i n} = e^z$
\begin{align*}
	e^{z + 2\pi i n} &= e^z * e^{2\pi i n}\\
	&= e^z * \left(e^{2\pi i}\right)^n\\
	&= e^z \left(\cos{2\pi} + i\sin{2\pi}\right)^n\\
	&= e^z \left(1 + 0\right)^n\\
	&= e^z\\
	&\qed
\end{align*}

\subsection{d)}
Compute $\operatorname{Re}{(\ldots)}$, $\operatorname*{Im}{(\ldots)}$ of $e^{1+\pi i}, e^{3+i\pi/3}, e^{5+i\pi/4}$
\begin{align*}
	e^{x+iy} &= e^x e^{iy}\\
	&= e^x \left( \cos{y} + i\sin{y}\right)\\
	\operatorname{Re}{(e^{x+iy})} &= e^x \cos{y}\\
	\operatorname{Im}{(e^{x+iy})} &= e^x \sin{y}
\end{align*}
\begin{enumerate}
	\item $z = 1+i\pi \implies \operatorname{Re}{(e^z)} = -e, \operatorname{Im}{(e^z)} = 0$
	\item $z = 3+i\pi/3 \implies \operatorname{Re}{(e^z)} = e^3/2, \operatorname{Im}{(e^z)} = e^3  * \sqrt{3}/2$
	\item $z = 5+i\pi/4 \implies \operatorname{Re}{(e^z)} = e^5 * \sqrt{2}/2, \operatorname{Im}{(e^z)} = e^5 * \sqrt{2}/2$
\end{enumerate}
\newpage
\section{}\hrule
\subsection{a)}
\begin{align*}
	\cos{(x)} &= \frac{e^{ix}+e^{-ix}}{2}\\
	&= \frac{\cos{(x)} + i\sin{(x)} + \cos{(-x)} + i\sin{(-x)} }{2}\\
	&= \frac{2\cos{(x)}}{2} \tag*{$\cos{(-x)} = \cos{(x)}, \sin{(-x)} = \sin{(x)}$}\\
	&= \cos{(x)}\\
	&\qed
\end{align*}
\begin{align*}
	\sin{(x)} &= \frac{e^{ix}-e^{-ix}}{2i}\\
	&= \frac{\cos{(x)} + i\sin{(x)} - \cos{(-x)} - i\sin{(-x)} }{2i}\\
	&= \frac{2i\sin{(x)}}{2i} \tag*{$\cos{(-x)} = \cos{(x)}, \sin{(-x)} = \sin{(x)}$}\\
	&= \sin{(x)}\\
	&\qed
\end{align*}
\subsection{b)}
\begin{align*}
	\cos{(2 + 3i)} &= \frac{e^{2+3i} + e^{-2-3i}}{2}\\
	&= (1/2) \left(e^2(\cos{(3)} + i\sin{(3)}) + e^{-2}(\cos{(-3)} + i\sin{(-3)})\right)\\
	&= (1/2) \left(e^2\cos{(3)} + ie^2\sin{(3)} + e^{-2}\cos{(-3)} + ie^{-2}\sin{(-3)}\right)\\
	&= (1/2) \left( e^2\cos({3}) +e^{-2}\cos{(-3)} + i\left(e^2\sin{(3)} + e^{-2}\sin{(-3)}\right) \right)
\end{align*}
\begin{align*}
	\sin{(2 + 3i)} &= \frac{e^{2+3i} - e^{-2-3i}}{2i}\\
	&= (1/2i) \left(e^2(\cos{(3)} + i\sin{(3)}) - e^{-2}(\cos{(-3)} - i\sin{(-3)})\right)\\
	&= (1/2i) \left(e^2\cos{(3)} + ie^2\sin{(3)} - e^{-2}\cos{(-3)} - ie^{-2}\sin{(-3)}\right)\\
	&= (1/2i) \left( e^2\cos({3}) -e^{-2}\cos{(-3)} - i\left(e^2\sin{(3)} + e^{-2}\sin{(-3)}\right) \right)
\end{align*}
\subsection{c)}
\begin{align*}
	\cos{(z)} &= \cos{(x+iy)}\\
	&= \frac{e^{i(x+iy)} + e^{i(-x-iy)}}{2}\\
	&= \frac{e^{ix-y} + e^{y-ix}}{2}\\
	&= \frac{e^{-y}\cos{(x)} + ie^{-y}\sin{(x)} + e^{y}\cos{(-x)} + ie^{y}\sin{(-x)}}{2}\\
	&= \frac{e^{-y}\cos{(x)} + e^{y}\cos{(-x)} + ie^{-y}\sin{(x)} + ie^{y}\sin{(-x)}}{2}\\
	&= \frac{e^{-y}\cos{(x)} + e^{y}\cos{(x)} + ie^{-y}\sin{(x)} - ie^{y}\sin{(x)}}{2}\\
	&= \frac{\cos{(x)}\left(e^{-y}+ e^{y}\right) + i\sin{(x)}\left(e^{-y} - e^{y}\right)}{2}\\
	&= \frac{\cos{(x)}\left(e^{-y}+ e^{y}\right) - i\sin{(x)}\left(e^{y} - e^{-y}\right)}{2}\\
	&= \cos{(x)}\cosh{(y)} - i\sin{(x)}\sinh{(y)}\\
	&\qed
\end{align*}
\subsection{d)}
\begin{align*}
	\sin{(z)} &= \sin{(x+iy)}\\
	&= \frac{e^{i(x+iy)} - e^{i(-x-iy)}}{2i}\\
	&= \frac{e^{ix-y} - e^{y-ix}}{2i}\\
	&= \frac{e^{-y}\cos{(x)} + ie^{-y}\sin{(x)} - e^{y}\cos{(-x)} - ie^{y}\sin{(-x)}}{2i}\\
	&= \frac{e^{-y}\cos{(x)} - e^{y}\cos{(-x)} + ie^{-y}\sin{(x)} - ie^{y}\sin{(-x)}}{2i}\\
	&= \frac{e^{-y}\cos{(x)} - e^{y}\cos{(x)} + ie^{-y}\sin{(x)} + ie^{y}\sin{(x)}}{2i}\\
	&= \frac{\cos{(x)}\left(e^{-y} - e^{y}\right) + i\sin{(x)}\left(e^{-y} + e^{y}\right)}{2i}\\
	&= \frac{\cos{(x)}\left(e^{-y} - e^{y}\right) + i\sin{(x)}\left(e^{y} + e^{-y}\right)}{2i} * \frac{i}{i}\\
	&= \frac{-i\cos{(x)}\left(e^{-y} - e^{y}\right) + \sin{(x)}\left(e^{y} + e^{-y}\right)}{2}\\
	&= \frac{i\cos{(x)}\left(e^{y} - e^{-y}\right) + \sin{(x)}\left(e^{y} + e^{-y}\right)}{2}\\
	&= i\cos{(x)}\sinh{(y)} + \sin{(x)}\cosh{(y)}\\
	&\qed
\end{align*}
\newpage
\section{}\hrule
\subsection{a)}
\begin{align*}
	\cos{(-z)} &= \frac{e^{-iz} + e^{--iz}}{2}\\
	&= \frac{e^{iz} + e^{-iz}}{2}\\
	&= \cos{(z)}\\
	&\qed
\end{align*}
\begin{align*}
	\sin{(-z)} &= \frac{e^{-iz} - e^{--iz}}{2}\\
	&= -\frac{e^{iz} - e^{-iz}}{2}\\
	&= -\sin{(z)}\\
	&\qed
\end{align*}
\subsection{b)}
\begin{align*}
	\cos{(\pi/2 - z)} &= \frac{e^{i\pi/2 - iz} + e^{-i\pi/2 + iz}}{2}\\
	&= \frac{\dfrac{e^{i\pi/2}}{e^{iz}} + \dfrac{e^{iz}}{e^{i\pi/2}}}{2}\\
	&= (1/2) \dfrac{e^{i\pi} + e^{2iz}}{e^{iz+i\pi/2}}\\
	&= (1/2) \left(\dfrac{e^{i\pi}}{e^{iz+i\pi/2}} +\dfrac{e^{2iz}}{e^{iz+i\pi/2}} \right)\\
	&= (1/2) \left(e^{-iz+i\pi/2} + e^{iz-i\pi/2}\right)\\
	&= (1/2) \left(e^{-iz}e^{i\pi/2} + e^{iz}e^{-i\pi/2}\right)\\
	&= (1/2) \left(ie^{-iz} - ie^{iz}\right) \tag{$e^{i\pi/2} = i, e^{-i\pi/2} = -i$}\\
	&= \dfrac{1}{2} \dfrac{i}{i} \left(ie^{-iz} - ie^{iz}\right)\\
	&= \dfrac{1}{2i} \left(-e^{-iz} + e^{iz}\right)\\
	&= \dfrac{1}{2i} \left(e^{iz} - e^{-iz}\right)\\
	&= \sin{(z)}\\
	&\qed
\end{align*}
\begin{align*}
	\sin{(\pi/2-z)} &= \dfrac{e^{i\pi/2-iz} - e^{-i\pi/2 + iz}}{2i}\\
	&= \dfrac{e^{i\pi/2}/e^{iz}-e^{iz}/e^{i\pi/2}}{2i}\\
	&= \dfrac{\dfrac{e^{i\pi} - e^{2iz}}{e^{iz+i\pi/2}}}{2i}\\
	&= \dfrac{e^{-iz+i\pi/2} - e^{iz-i\pi/2}}{2i}\\
	&= \dfrac{ie^{-iz} + ie^{iz}}{2i}\\
	&= \dfrac{e^{-iz} + e^{iz}}{2}\\
	&= \cos{(z)}\\
	&\qed
\end{align*}
\subsection{c)}
\begin{align*}
	\sin^2{(z)} + \cos^2{(z)} &= \left( \dfrac{e^{iz} - e^{-iz}}{2i}\right) ^2 + \left(\dfrac{e^{iz} + e^{-iz}}{2}\right)^2\\
	&= \dfrac{e^{2iz} - 2 + e^{-2iz}}{-4} + \dfrac{e^{2iz} + 2 + e^{-2iz}}{4}\\
	&= \dfrac{\cancel{e^{2iz}} - 2 + \cancel{e^{-2iz}}}{-4} + \dfrac{\cancel{e^{2iz}} + 2 + \cancel{e^{-2iz}}}{4}\\
	&= \dfrac{1}{2} + \dfrac{1}{2}\\
	&= 1\\
	&\qed
\end{align*}
\subsection{d)}
\begin{align*}
	\cos{(2z)} &= \dfrac{e^{2iz} + e^{-2iz}}{2}\\
	&= 1 - 2\sin{(z)}^2\\
	&= 1 - 2 \left(\dfrac{e^{iz} - e^{-iz}}{2i}\right)^2\\
	&= 1 - 2 \left(\dfrac{e^{2iz} - 2 + e^{-2iz}}{-4}\right)\\
	&= 1 + \left(\dfrac{e^{2iz} - 2 + e^{-2iz}}{2}\right)\\
	&= \cancel{1} + \left(\dfrac{e^{2iz} - \cancel{2} + e^{-2iz}}{2}\right)\\
	&= \dfrac{e^{2iz} + e^{-2iz}}{2}
	&= \cos{(2z)}\\
	&\qed
\end{align*}
\newpage
\section{}\hrule
Let $n\in\mathbb{Z}$

$z = 1 + i\sqrt{3} = 2e^{i\pi/3}$

$w = -1 + i = \sqrt{2}e^{i3\pi/4}$

$\operatorname{Log}{(z)} = \operatorname{Log}{(2)} + i\pi/3$

$\log{(z)} = \operatorname{Log}{(z)} + 2\pi i n$

$\operatorname{Log}{(w)} = \operatorname{Log}{(\sqrt{2})} + i3\pi/4$

$\log{(w)} = \operatorname{Log}{(w)} + 2\pi i n$

The difference is that $\operatorname{Log}$ has an unique solution, while $\log$ has infinitely many solutions.
\newpage
\section{}\hrule
\begin{align*}
	i^{1+i} &= \left(e^{i\pi/2}\right)^{1+i}\\
	&= e^{i\pi/2 * (1+i)}\\
	&= e^{i\pi/2 - \pi/2}\\
	&= \dfrac{e^{i\pi/2}}{e^{\pi/2}}\\
	&= \dfrac{i}{e^{\pi/2}}\\
	&= ie^{-\pi/2}\\
	&\qed
\end{align*}
\newpage
\section{}\hrule
Let $ A = e^{iz} $

$\arctan{(2+i)}$

\begin{align*}
	\tan{(z)} &= 2+i\\
	\dfrac{\sin{(z)}}{\cos{(z)}} &= 2 + i\\
	\cancel{1/2 *\dfrac{1}{1/2}}  *1/i * \dfrac{e^{iz} - e^{-iz}}{e^{iz}+e^{-iz}} &= 2+i\\
	1/i \dfrac{A-A^{-1}}{A + A^{-1}} &= 2+i \tag{$A=e^{iz}$}\\
	\dfrac{A-A^{-1}}{A + A^{-1}} &= 2i-1\\
	A - A^{-1} &= (2i-1)\left(A+A^{-1}\right)\\
	A^2 - 1 &= (2i-1)\left(A^2+1\right)\\
	A^2 - 1 &= 2iA^2 + 2i -A^2 -1\\
	2A^2 -2iA^2 &= 2i\\
	A^2(2-2i) &= 2i\\
	A^2 &= \dfrac{i}{1-i}\\
	2 \log{(e^{iz})} &= \log{(i)} - \log{(1-i)}\\
	2iz &= \log{(e^{i\pi/2})} - \log{(\sqrt{2}e^{-i\pi/4})}\\
	&= i\pi/2 - \log{(\sqrt{2})} +i\pi/4\\
	z  &= \dfrac{i\pi/2 - \log{(\sqrt{2})} +i\pi/4}{2i}\\
	&= \dfrac{i3\pi/4 - \log{(\sqrt{2})}}{2i}\\
	&= \boxed{\dfrac{3\pi/4 + i\log{(\sqrt{2})}}{2}}
\end{align*}

$ \arccos{\left(\dfrac{e^2+e^{-2}}{2}\right)} $
\begin{align*}
	\cos{(z)} &= \dfrac{e^2+e^{-2}}{2}\\
	\dfrac{A + A^{-1}}{2} &= \tag{$A=e^{iz}$}\\
	A^2 + 1 &= A \left(e^2 + e^{-2}\right)\\
	A^2 - A\left(e^2 + e^{-2}\right) + 1 &= 0\\
	A &= e^2, e^{-2}\\
	e^{iz} &= \\
	iz &= 2, -2\\
	z &= \boxed{2i, -2i}
\end{align*}
\newpage
\section{}\hrule
\subsection{a)}
$\diffp{u}{x} = \diffp{v}{y} \land \diffp{u}{y} = -\diffp{v}{x}$ 
\subsection{b)}
$f(z)$ is analytic if $f'(z)$ exists.
\subsection{c)}
If the 4 partials are continuous and the Cauchy-Riemann equations are satisfied, then the function is analytic.
\subsection{d)}
$ f'(z) = u_x(x,y) + iv_x(x,y) $, where subscripts are the shorthand for partial differentiation.
\newpage
\section{}\hrule
\subsection{a)}
$ u(x,y) = u = e^x \cos{(y)}, v(x,y) = v = e^x \sin{(y)} $

$u_x = e^x \cos{(y)} = v_y, u_y = -e^x \sin{(y)} = -v_x$

$ f'(z) = u_x + iv_x = e^x\cos{(y)} + ie^x \sin{(y)} = e^z $

\subsection{b)}

$ u = \sin{(x)}\cosh{(y)}, v = \cos{(x)}\sinh{(y)}$

$u_x = \cos{(x)}\cosh{(y)} = v_y, u_y = \sin{(x)}\sinh{(y)} = -v_x$

$ f'(z) = u_x + iv_x = \cos{(x)}\cosh{(y)} + i\sin{(x)}\sinh{(y)} = \cos{(z)} $
\newpage
\section{}\hrule
\subsection{a)}
$ u_x = 3Ax^2 + By^2 + C = Dx^2 + 3Ey^2 + F = v_y, u_y = 2Bxy = -2Dxy = -v_x$

Since these 4 partials are polynomials with real coefficients, we know that they are continuous. After matching the terms up, we get $f(z)$ is analytic when $ 3A = D \land B = 3E \land C = F \land B = -D$.

\subsection{b)}
By setting $ y=0 \implies z = x $, we get 
\[ f(z) = Az^3 + Dz^2 + Bx + C\]

\subsection{c)}
\[f'(z) = 3Az^2 + 2Dz + B\]
\newpage
\section{}\hrule
\subsection{a)}
\[u_x = -e^{-y}\sin{(x)} = v_y, u_y = -e^{-y}\cos{(x)} = -v_x\]
These 4 partials are continuous because they are the product of 2 continuous functions.
\subsection{b)}
\[g'(z) = -e^{-y}\sin{(x)} + ie^{-y}\cos{(x)}\]
\subsection{c)}
\[g(z) = e^{-y+ix} = e^{i(x+iy)} = e^{iz}\]
\subsection{d)}

Verify $ u_{xx} + u_{yy} = 0 $

\[u_x = -e^{-y}\sin{(x)} \implies u_{xx} = -e^{-y}\cos{(x)}\]
\[u_y = -e^{-y}\cos{(x)} \implies u_{yy} = e^{-y}\cos{(x)}\]
\[u_{xx} + u_{yy} = 0\]

\end{document}


