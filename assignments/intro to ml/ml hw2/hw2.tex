\documentclass[12pt]{article}
\usepackage[utf8]{inputenc}
\usepackage[top=1in, bottom=0.75in, left=0.75in, right=0.75in, headheight=15pt]{geometry}
\usepackage{amsmath, amssymb, amsthm, graphicx, hyperref, enumerate, multirow,  multicol, tikz, centernot, cancel, forest, lipsum, mathtools, bm, esvect, fancyhdr, esdiff, float, parskip, comment}

\DeclareMathSymbol{*}{\mathbin}{symbols}{"01} % change * to /cdot inside math
% \begingroup % let only this align, etc. break across pages
% \allowdisplaybreaks
% \begin{align}
%     ....
% \end{align}
% \endgroup

% \texorpdfstring{$k$}{k} math inside (sub)section label

\pagestyle{fancy}
\fancyhead[L]{Liheng Cao}
% \fancyhead[C]{center}
\fancyhead[R]{CS4563}

\title{Title}
\author{Liheng Cao}
% \date{Date}

\begin{document}
\maketitle

\section{}
\begin{itemize}
	\item $X$ = 
	$\begin{bmatrix}
		1 & 0 & 0 \\
		1 & 0 & 1 \\
		1 & 1 & 0 \\
		1 & 1 & 1 
	\end{bmatrix}$

	\item $y$ = 
	$ \begin{bmatrix}
		1 \\
		4 \\
		3 \\
		7 \\
	\end{bmatrix} $

	\item The closed form solution is $ (X^TX)^{-1}X^Ty $
	
	\item $	\begin{bmatrix}
		w_0 \\
		w_1 \\
		w_2 \\ 
	\end{bmatrix} =
	(X^TX)^{-1}X^Ty = 
	\left(\begin{bmatrix}
		1 & 0 & 0 \\
		1 & 0 & 1 \\
		1 & 1 & 0 \\
		1 & 1 & 1 
	\end{bmatrix}^T
	\begin{bmatrix}
	1 & 0 & 0 \\
	1 & 0 & 1 \\
	1 & 1 & 0 \\
	1 & 1 & 1 
	\end{bmatrix}\right)^{-1}
	\begin{bmatrix}
	1 & 0 & 0 \\
	1 & 0 & 1 \\
	1 & 1 & 0 \\
	1 & 1 & 1 
	\end{bmatrix}^T
	\begin{bmatrix}
	1 \\
	4 \\
	3 \\
	7 \\
	\end{bmatrix} = 
	\begin{bmatrix}
		3/4 \\
		5/2 \\
		7/2 \\
	\end{bmatrix}$

	\item RSS = $\sum\limits_{i=1}^{4} \left(3/4 + 5/2x_1^{(i)} + 7/2x_2^{(i)} - y^{(i)}\right)^2$ \\
	$=\left(3/4 + 5/2*0 + 7/2*0 - 1\right)^2 + \ldots + \left(3/4 + 5/2 + 7/2 - 7 \right)^2 =0.25$
	
	\item TSS = $\sum\limits_{i=1}^{4} \left(y^{(i)} - \bar{y}^{(i)}\right)^2 =$ \\
	$= \left(1 - 3.75\right)^2 + \ldots + \left(7 - 3.75\right)^2 =18.75$
	
	\item $ R^2 = 1 - \dfrac{RSS}{TSS} = 1 - \dfrac{0.25}{18.75} = 0.9867$
	
	\item This is the same thing as $ R^2 $. 98.\% of the variance in $ y $ is explained by $ x $.
\end{itemize}
\newpage
\section{}
$ w = \begin{bmatrix*}
	8.29\ldots & -0.66\ldots & -0.90\ldots & 4.58\ldots & -0.24\ldots\\
\end{bmatrix*}^T , N = 506$
\begin{itemize}
	\item RSS = $\sum\limits_{i=1}^{N} \left(w^T * X^{(i)} - y^{(i)}\right)^2 = 14517.66\ldots$
	
	\item TSS = $\sum\limits_{i=1}^{4} \left(y^{(i)} - \hat{y}^{(i)}\right)^2 = 42716.30\ldots$
	
	\item $ R^2 = 1 - \dfrac{14517.66\ldots}{42716.30\ldots} = 1 - \dfrac{0.25}{18.75} = 0.6601$
	
	\item 66.01\% of the variance in $ y $ can be explained by $ x $.
\end{itemize}
\newpage
\section{}
I would put the actual crop yields in a column vector $ y $, and then I would put the amount of rainfall, the amount of fertilizer, the average temperature, and the number of sunny days into a matrix $ X $. This matrix would have the first column be composed entirely of ones for the constant ``trick". If we use a linear model, then we can find the optimal solution using $ w = (X^TX)^{-1}X^Ty $.
\newpage
\section{}
$ w = (w_1, w_2) $
\begin{itemize}
	\item $ f_1 $
	\begin{enumerate}
		\item (7.0, 17.0)
		\item (-85.0, -222.0)
		\item (1202.0, 3106.0)
		\item (-16725.0, -43265.0)
	\end{enumerate}

	\item $ f_1 $
	\begin{enumerate}
		\item (15.0, 20.0)
		\item (-95.0, -460.0)
		\item (2045.0, 10060.0)
		\item (-44315.0, -219420.0)
	\end{enumerate} 
\end{itemize}

\end{document}


