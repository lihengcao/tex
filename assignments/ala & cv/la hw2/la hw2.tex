\documentclass[12pt]{article}
\usepackage[utf8]{inputenc}
\usepackage[top=0.75in, bottom=0.75in, left=0.75in, right=0.75in, headheight=15pt]{geometry}
\usepackage{amsmath, amssymb, amsthm, graphicx, hyperref, enumerate, multirow,  multicol, tikz, centernot, cancel, forest, lipsum, mathtools, bm, esvect, fancyhdr, esdiff, float, parskip, comment}

\DeclareMathSymbol{*}{\mathbin}{symbols}{"01} % change * to /cdot inside math
% \begingroup % let only this align, etc. break across pages
% \allowdisplaybreaks
% \begin{align}
%     ....
% \end{align}
% \endgroup

% \begin{figure}[H]
%     \centering
%     \includegraphics{}
%     \caption{}
%     \label{fig:}
% \end{figure}

% \texorpdfstring{$k$}{k} math inside (sub/)section label

\pagestyle{fancy}
\fancyhead[L]{Liheng Cao}
% \fancyhead[C]{center}
\fancyhead[R]{}

\title{} % title
\author{Liheng Cao} % name
\date{\today} % custom date else today's date

\begin{document}
\maketitle

\section{}
\begin{enumerate}[(a)]
	\item 
	\[ V = 
		\begin{bmatrix}
			1 & -4\\
			1 & 3
		\end{bmatrix}
		, D = 
		\begin{bmatrix}
			5 & 0\\
			0 & -2
		\end{bmatrix}
	\]
		 
	\item \[ Av_1 = v_1\lambda_1, Av_2 = v_2\lambda_2 \implies A [v_1, v_2] = [v_2, v_2]
		\begin{bmatrix}
			\lambda_1 & 0\\
			0 & \lambda_2
		\end{bmatrix}  
	\]
	
	\item 
		\begin{align*}
		A &= VDV^{-1}\\
		\begin{bmatrix}
			1 & 4 \\
			3 & 2
		\end{bmatrix}
			&=
			\begin{bmatrix}
				1 & -4\\
				1 & 3
			\end{bmatrix}
			\begin{bmatrix}
				5 & 0\\
				0 & -2
			\end{bmatrix}
			\begin{bmatrix}
				1 & -4\\
				1 & 3
			\end{bmatrix}^{-1}\\
		\begin{bmatrix}
			1 & 4 \\
			3 & 2
		\end{bmatrix}
			\begin{bmatrix}
				1 & -4\\
				1 & 3
			\end{bmatrix}
			&=
			\begin{bmatrix}
				1 & -4\\
				1 & 3
			\end{bmatrix}
			\begin{bmatrix}
				5 & 0\\
				0 & -2
			\end{bmatrix}\\
		\begin{bmatrix}
			5 & 8\\
			5 & -6
		\end{bmatrix}
			&= \begin{bmatrix}
				5 & 8\\
				5 & -6
			\end{bmatrix}
		\end{align*}
	
	\item The transformation matrix applied to one of it's eigenvectors is the same as multiplying the eigenvector by its respective eigenvalue.
	
	\item 
		\begin{align*}
			A^{33} &= VD^{33}V^{-1}\\
			&= 
				\begin{bmatrix}
					1 & -4\\
					1 & 3
				\end{bmatrix}
				\begin{bmatrix}
					5 & 0\\
					0 & -2
				\end{bmatrix}^{33}
				\begin{bmatrix}
					1 & -4\\
					1 & 3
				\end{bmatrix}^{-1}\\
			&= 
				\begin{bmatrix}
					1 & -4\\
					1 & 3
				\end{bmatrix}
				\begin{bmatrix}
					5^{33} & 0\\
					0 & (-2)^{33}
				\end{bmatrix}
				\begin{bmatrix}
					1 & -4\\
					1 & 3
				\end{bmatrix}^{-1}\\
			&= 
				\begin{bmatrix}
					1 & -4\\
					1 & 3
				\end{bmatrix}
				\begin{bmatrix}
					5 & 0\\
					0 & -2
				\end{bmatrix}^{33}
				\begin{bmatrix}
					1 & -4\\
					1 & 3
				\end{bmatrix}^{-1}\\
			&= 
				\dfrac{1}{7} \left( 5^{33}
				\begin{bmatrix}
					3 & 4\\
					3 & 4
				\end{bmatrix}
				+ (-2)^{33}
				\begin{bmatrix}
					4 & -4\\
					-3 & 3
				\end{bmatrix}
				\right)
		\end{align*}
\end{enumerate}
\newpage

\section{}
\begin{enumerate}[(a)]
	\item 
\end{enumerate}


\end{document}