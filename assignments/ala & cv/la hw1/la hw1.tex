\documentclass[12pt]{article}
\usepackage[utf8]{inputenc}
\usepackage[top=0.75in, bottom=0.75in, left=0.75in, right=0.75in, headheight=15pt]{geometry}
\usepackage{amsmath, amssymb, amsthm, graphicx, hyperref, enumerate, multirow,  multicol, tikz, centernot, cancel, forest, lipsum, mathtools, bm, esvect, fancyhdr, esdiff, float, parskip, comment}

\DeclareMathSymbol{*}{\mathbin}{symbols}{"01} % change * to /cdot inside math
% \begingroup % let only this align, etc. break across pages
% \allowdisplaybreaks
% \begin{align}
%     ....
% \end{align}
% \endgroup

% \begin{figure}[H]
%     \centering
%     \includegraphics{}
%     \caption{}
%     \label{fig:}
% \end{figure}

% \texorpdfstring{$k$}{k} math inside (sub/)section label

\pagestyle{fancy}
\fancyhead[L]{Liheng Cao}
% \fancyhead[C]{center}
\fancyhead[R]{}

\title{} % title
\author{Liheng Cao} % name
\date{\today} % custom date else today's date

\begin{document}
\maketitle

\section{}
\begin{enumerate}
	\item [Matrix A]
	\begin{enumerate}[(a)]
		\item \[ \lambda_1 = 4, \lambda_2 = -2, v_1 = (1,1), v_2 = (-1,1)\]
		
		\item \[ 1+1 = 1+1 = 2 \]
		
		\item \[ 4*-2 = -8 \]
		
		\item \[ \arccos\left(\dfrac{\langle(1,1),(-1,1)\rangle}{\sqrt{1+1}*\sqrt{1+1}}\right) = \arccos\left(0\right) = \pi/2 \]
		
		\item The eigenvalues are real. The matrix is symmetric.
	\end{enumerate}

	\item [Matrix B]
	\begin{enumerate}[(a)]
		\item \[ \lambda_1 = 37, \lambda_2 = -15, v_1 = (3,2), v_2 = (-2,3)\]
		
		\item \[ 37-15 = 21+1 = 22 \]
		
		\item \[ 37*-15 = -555 \]
		
		\item \[ \arccos\left(\dfrac{\langle(3,2),(-2,3)\rangle}{\sqrt{3^2+2^2}*\sqrt{3^2+2^2}}\right) = \arccos\left(0\right) = \pi/2 \]
		
		\item The eigenvalues are real. The matrix is symmetric.
	\end{enumerate}

	\item [Matrix C]
	\begin{enumerate}[(a)]
		\item \[ \lambda_1 = -1, \lambda_2 = 1, \lambda_3 = 1, v_1 = (-1,0,1), v_2 = (1,0,1), v_3 = (0,1,0)\]
		
		\item \[ -1+1+1 = 0+1+0 = 1 \]
		
		\item \[ -1*1*1 = -1 \]
		
		\item Find the inner products. $ \langle v_1, v_2 \rangle = \langle v_2, v_3 \rangle = \langle v_1, v_3 \rangle = 0 $. Since the inner products are all 0, then the angle is $ \pi/2 $
		
		\item The eigenvalues are real. The matrix is symmetric.
	\end{enumerate}

	\item [(f)]
	Given that $ X $ has all real values, $ X = X^T \implies \operatorname{Im}(v_i) = 0$ 
\end{enumerate}
\newpage

\section{}
\[ \theta = \arccos\left(\dfrac{\langle v, w\rangle}{\|v\| \|w\|}\right) \]

\[ \mathit{PQ} = (2,1,0,3) - (1,2,0,1) = (1,-1,0,2) \]
\[ \mathit{PR} = (0,1,1,0) - (1,2,0,1) = (-1,-1,1,-1) \]
\[ \mathit{RQ} = (2,1,0,3) - (0,1,1,0) = (2,0,-1,3) \]

As long as we take the acute angle, it doesn't matter if a vector is backwards: \textit{PQ}, \textit{QP} are equivalent.

\[ \textit{P} = \arccos\left(\dfrac{\langle \mathit{PQ}, \mathit{PR}\rangle}{\|\mathit{PQ}\| \|\mathit{PR}\|}\right) = 114.095^{\circ}\]

\[ \textit{Q} = \arccos\left(\dfrac{\langle \mathit{PQ},\mathit{RQ}\rangle}{\|\mathit{PQ}\| \|\mathit{RQ}\|}\right) = 29.206^{\circ}\]

\[ \textit{R} = 180-\arccos\left(\dfrac{\langle\mathit{PR},\mathit{RQ}\rangle}{\|\mathit{PR}\| \|\mathit{RQ}\|}\right) = 36.700^{\circ}\]

The 3 sum to $ 180^{\circ} $.
\newpage

\section{}
\begin{align*}
	\langle \vec{N}, \vec{X}-P\rangle &= 0\\
	\langle\vec{N}, \vec{X} \rangle - \langle \vec{N}, P\rangle &= 0\\
	(a_1x_1 + \cdots + a_nx_n) - (r_1x_1 + \cdots + c_nx_n) &= 0\\ \tag{solves means evals to $b$}
	b - b &= 0
\end{align*}
\newpage

\section{}
\[ \theta = \arccos\left(\dfrac{\langle (1,1,1,1), (2,1,3,7)\rangle}{\sqrt{1 + 1 + 1 + 1}*\sqrt{2^2 + 1 + 3^2 + 7^2}}\right) = 35.023^{\circ} \]
\newpage

\section{}
\[ \theta = \arccos\left(\dfrac{\langle (1,1,1,1), (2,3,0,5)\rangle}{\sqrt{1 + 1 + 1 + 1}*\sqrt{2^2+3^2+0+5^2}}\right) = 35.800^{\circ} \]
\newpage

\section{}
\begin{enumerate}[(a)]
	\item \,
	\begin{figure}[H]
		\centering
		\includegraphics[width=\textwidth]{images/cube.png}
		\caption{rough sketch}
		\label{fig:a:sketch}
	\end{figure}

	\item \[ \sqrt{1+1+1} = \sqrt{3} \]
	
	\item \[ \arccos\left(\dfrac{\langle (1,0,0), (1,1,1)\rangle}{\sqrt{1} * \sqrt{3}}\right) = 54.736^{\circ}\]
\end{enumerate}
\newpage

\section{}
\begin{enumerate}[(a)]
	\item \[ \|x^2\| = \sqrt{\langle x^2, x^2 \rangle} = \sqrt{\int_{0}^{1} x^2 * x^2 \,dx} = \sqrt{1/5}\]
	\[ \|x^3\| = \sqrt{\langle x^3, x^3 \rangle} = \sqrt{\int_{0}^{1} x^3 * x^3 \,dx} = \sqrt{1/7}\]
	\[\theta = \arccos\left(\dfrac{\langle x^2, x^3 \rangle}{\sqrt{1/35}}\right) = \arccos\left(\dfrac{\int_{0}^{1} x^2 x^3 \,dx}{\sqrt{1/35}}\right) = 9.594^{\circ}\]
	
	\item \[ \|\sin(m\pi x)\| = \sqrt{\langle \sin(m\pi x), \sin(m\pi x) \rangle} = \sqrt{\int_{0}^{1} \sin(m\pi x)\sin(m\pi x)\, dx} = \dfrac{1}{2} - \dfrac{\sin(2m\pi )}{4\pi^2}\]
	
	\[\|\sin(n\pi x)\|  \]
\end{enumerate}


\end{document}